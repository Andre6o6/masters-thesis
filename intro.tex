% !TeX spellcheck = ru_RU

Огромное количество информации об окружающем мире человеческий мозг воспринимает интуитивно. 
От классификации объектов среди десятков тысяч возможных вариантов, до распознавания трехмерных сцен, детектирования движения и взаимодействий между объектами —-- человеческий мозг способен извлечь и проанализировать эту информацию за доли секунд \cite{dicarlo2012brain}. 
Построение компьютерных алгоритмов, способных приблизиться по производительности и гибкости к механизмам распознавания в человеческом мозге является одной из первостепенных задач области под названием Computer Vision.

Эффективным инструментом извлечения и анализа такой информации являются генеративные модели \cite{jebara2012machine}. 
Их основная идея отражена в известной цитате Ричарда Фейнмана: “Я не понимаю то, что я не могу воссоздать”. 
Эти модели нацелены на моделирование наблюдаемых данных, выявление их внутренней структуры и использование этой структуры для генерации новых данных.

Среди генеративных моделей, используемых в Computer Vision, на сегодняшний день генеративные состязательные сети показывают  лучшие результаты \cite{generative2020survey}. 
Генеративные состязательные \cite{goodfellow2014generative} сети формулируют процесс обучения генеративной модели в виде антагонистической игры между двумя игроками --- генератором и дискриминатором. 
Генератор является нейронной сетью, задача которой состоит в генерировании синтетических данных, как можно более похожих на реальные. 
Дискриминатор также является нейронной сетью, задача которой состоит в том, чтобы как можно более точно отличать синтетические данные, полученные генератором, от реальных данных, взятых из обучающей выборки. 
Процесс совместного конкурентного обучения этих сетей продолжается вплоть до достижения системой равновесия по Нэшу \cite{goodfellow2017nips}, в котором генератор способен генерировать изображения, неотличимые дискриминатором от реальных.

Особенно большие успехи были достигнуты генеративными состязательными сетями в задаче генерации изображений лиц \cite{progressive-growing-gan, StyleGAN, karras2020stylegan2}. 
К этой задаче проявляется особый интерес, т.к. она хорошо демонстрирует возможности генеративных состязательных сетей по кодированию визуальной информации. 
Ведь известно, что лица обладают большим количеством факторов вариации, которые легко и интуитивно распознаются человеком, но при этом трудно поддаются анализу машинными алгоритмами \cite{face2008perception}.
Генеративные состязательные сети способны упаковать информацию о факторах вариации в виде вектора небольшой длины, а затем восстановить из этого вектора исходное изображение, сохраняя  высокую реалистичность и разнообразие признаков. 
Стоит также отметить, что последние нейробиологические исследования указывают на наличие схожих механизмов кодирования информации о лицевых признаках в мозге человека \cite{facial2017code}.

Существующие генеративные состязательные сети способны генерировать реалистичные изображения лиц высокого разрешения, т.к. они обучились эффективно кодировать информацию о признаках (другими словами \emph{семантику}). 
Можно предположить, что такое кодирование позволяет генеративным состязательным сетям также хорошо справляться с задачей редактирования изображений в терминах признаков лиц, которые интуитивно понятны человеку (далее --- \emph{задача семантического редактирования}). 
Тем не менее, существующие подходы, которые применяют генеративные состязательные сети для решения этой задачи, не могут оперировать всем многообразием признаков, которое доступно при генерации изображений. 
Большинство таких подходов ограничивается редактированием единственного признака, а для работы с другими признаками они требуют полного переобучения \cite{cycle2017gan, residual2017gan}. 
В других случая они  модифицируют  архитектуру генеративной состязательной сети, позволяя задавать лишь определенное значение для фиксированного набора признаков \cite{choi2018stargan, liu2019stgan}. 
А это, в свою очередь, не позволяет им в полной мере использовать извлекаемую при обучении семантику изображений, а также усложняет внедрение новых техник и улучшений, появляющихся в области генерации изображений.

В данной работе предлагается решить задачу семантического редактирования с помощью генеративных состязательные сетей, взяв в качестве основы предобученную сеть для генерации лиц, и модифицировать ее архитектуру для предоставления возможности манипулировать ее внутренними представлениями.
