Огромное количество информации об окружающем мире человек воспринимает интуитивно.

Несмотря на успехи в машинном и глубоком обучении, дискриминативные модели, т.е. модели, которые по наблюдаемым данным могут предсказать какую-то метку класса или какое-то значение, не могут понять сами данные. А потому эта информация им недоступна.

“Я не понимаю то, что я не могу воссоздать” — эта известная цитата Ричарда Фейнмана отражает основную идею генеративных моделей \cite{jebara2012machine}. Эти модели пытаются смоделировать данные, которые они наблюдают.

На сегодняшний день волну развития в области генеративных моделей возглавляют генеративные состязательные сети \cite{goodfellow2014generative}. Они представляют собой две нейронные сети, которые обучаются в тандеме: первая по входному случайному вектору учится генерировать изображение, вторая по сгенерированному изображению учится определять — реальное оно или нет?

Особенно большие успехи были достигнуты генеративными состязательными сетями в задаче генерации изображений лиц [progressive-gan, stylegan, stylegan2]. Эта задача обладает особой теоретической важностью, т.к. человеческий мозг способен интуитивно распознавать лица с малых лет и замечает их малейшие вариации и неточности.

Существующие генеративные состязательные сети способны генерировать реалистичные изображения лиц высокого разрешения, т.к. они обучились эффективно кодировать скрытую структуру изображений в своих внутренних представлениях.
Можно предположить, что генеративные состязательные сети также будут хорошо справляться с задачей семантического редактирования изображений, т.е. редактирования изображений в терминах признаков лиц, которые интуитивно понятны человеку.
Тем не менее, у подходов, которые напрямую обучают генеративные состязательные сети для решения этой задачи есть существенные недостатки.

В данной работе предлагается решить задачу семантического редактирования, взяв в качестве основы предобученную генеративную состязательную сеть для генерации лиц, и модифицировав ее архитектуру для предоставления возможности манипулировать ее внутренними представлениями.