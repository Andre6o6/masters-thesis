% !TeX spellcheck = ru_RU

\begin{figure}[h]
\begin{center}
    \includegraphics[width=0.9\textwidth]{Architecture_eng}
    \caption{Архитектура решения.}
    \label{fig:architecture}
\end{center}
\end{figure}


В рамках данной работы было разработано решение для семантического редактирования изображений, позволяющее редактировать изображения лиц в терминах высокоуровневых лицевых признаков, таких как положение головы, наличие улыбки и т.д.
По заданному бинарному лицевому признаку, это решение способно выделить семантический вектор в латентном пространстве генеративной состязательной сети, и произвести перенос информации об этом признаке с изображения-образца на целевое изображение.

Решение состоит из двух функциональных частей: модуля реконструкции изображений (рис. \ref{fig:architecture}, \emph{Image reconstruction module}), и модуля семантического редактирования (рис. \ref{fig:architecture}, \emph{Feature transfer module}).

\subsection{Модуль реконструкции изображений}
Модуль реконструкции изображений осуществляет отображение входных изображений в латентное пространство имеющейся генеративной состязательной сети.
Он реализует гибридный алгоритм отображения, совмещающий обучение энкодера и латентную оптимизацию. 
Энкодер (компонент \emph{Initial predictor} на рис. \ref{fig:architecture}) используется для получения грубого начального приближения.
Затем это приближение итеративно улучшается с помощью латентной оптимизации (компонент \emph{Latent optimizer} на рис. \ref{fig:architecture}).

Такой алгоритм предостравляет пользователю гибкий контроль баланса между точностью реконструкции изображения и затраченным на обработку временем.

В качестве используемой архитектуры генеративной состязательной сети выбрана сеть StyleGAN, что позволет производить оптимизацию в расширенном латентном пространстве $\mathcal W+$.

\subsection{Модуль семантического редактирования}
Модуль семантического редактирования осуществляет семантическое редактирование изображений, полученных в результате реконструкции входных изображений в латентном пространстве генеративной состязательной сети. 

Пользователю предоставляется возможность выбрать в качестве редактируемого признака один из признаков, для которых семантические вектора были вычислены заранее.
Альтернативно, данный модуль предоставляет возможность извлечь семантический вектор для произвольного бинарного лицевого признака.
Для этого модулем реализуется алгоритм InterFaceGAN (компонент \emph{Semantic vector extraction} на рис. \ref{fig:architecture}).

Найденный семантический вектор передается в компонент, реализующий алгоритм переноса признака с изображения-образца на целевое изображение (\emph{Feature transfer} на рис. \ref{fig:architecture}).
Этот алгоритм основывается на алгоритме линейного сдвига латентного вектора, описанном в \cite{abdal2019image2stylegan}.
Для того, чтобы добиться более качественного переноса редактируемого признака и сохранения остальных признаков, производится модификация данного алгоритма, позволяющая производить нелинейный сдвиг в латентном пространстве сети.

Перенос признака сводится к минимизации функции потерь переноса
$$ 
\min_{\mathbf w} L(\mathbf w) = L_{feature}(\mathbf w, \mathbf w_{exemplar}) + \alpha~L_{identity}(\mathbf w, I_{real}),
$$
где член $L_{feature}$ осуществляет линейный сдвиг латентного вектора, найденного в ходе реконструкции входного изображения, в направлении, заданном семантическим вектором признака;  член $~L_{identity}$ осуществляет корректировку сдвига таким образом, чтобы он изменял лишь рассматриваемый признак, и, в частности, не затрагивал признаки, определяющие личность изображенного человека.
