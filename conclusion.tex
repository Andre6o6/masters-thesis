В рамках данной выпускной квалификационной работы разработано\footnote{https://github.com/Andre6o6/stylegan-editing} решение для редактирования лицевых признаков на изображении, использующее предобученную для генерации изображений лиц генеративную состязательную сеть.

В ходе работы были получены следующие результаты:

\begin{itemize}
\item Проведен обзор существующих архитектур генеративных состязательных сетей для генерации изображений лиц --- BigGAN, StyleGAN, ALAE --- и приведены доводы в пользу использования архитектуры StyleGAN в качестве основы для алгоритма.
\item Реализован алгоритм выделения линейных подпространств, соответствующих семантикам изображения, в латентном пространстве сети.
\item Разработана архитектура нейронной сети для отображения входного изображения в латентное пространство, позволяющая контролировать баланс между точностью реконструкции изображения и затраченным временем.
\item Разработан алгоритм переноса произвольного бинарного лицевого признака с изображения-образца на целевое изображение. 
\item Прототип решения реализован на языке Python с использованием фреймворка PyTorch.
\item Проведены эксперименты для оценки работы алгоритма. Результаты проведенных экспериментов показывают улучшение качества реконструкции реальных изображений по сравнению с существующими подходами --- StyleGAN2 Projector и ALAE. В эксперименте по реконструкции изображений из датасета FFHQ предложенное решение показывает улучшение метрик визуальной схожести LPIPS и Face Identity Score.
\item Также результаты проведенных экспериментов показывают улучшение переноса лицевых признаков по сравнению с существующими подходами --- Image2StyleGAN и GANSpace.  В эксперименте по переносу признаков на изображениях из датасета FEI Face Database предложенное решение показывает улучшение метрики реалистичности FID, а также LPIPS и Face Identity Score.
\end{itemize}