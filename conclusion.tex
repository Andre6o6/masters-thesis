% !TeX spellcheck = ru_RU

В рамках данной выпускной квалификационной работы разработано решение для редактирования лицевых признаков на изображении, использующее предобученную для генерации изображений лиц генеративную состязательную сеть StyleGAN. 

В ходе работы были получены следующие результаты:

\begin{itemize}
\item Проведен обзор существующих архитектур генеративных состязательных сетей для генерации изображений лиц --— BigGAN, StyleGAN, ALAE; приведены доводы в пользу использования архитектуры StyleGAN в качестве основы для решения.
\item Разработана архитектура решения для семантического редактирования изображений, позволяющая произвести перенос произвольного бинарного лицевого признака с изображения-образца на целевое изображение.
\item Реализован прототип решения на языке Python с использованием фреймворка PyTorch. Исходные коды выложены на GitHub\footnote{https://github.com/Andre6o6/stylegan-editing}.

% TODO: найти не кривой способ разобраться с переносом англ. слов!
\item Проведены эксперименты для оценки работы решения. 
Результаты показывают улучшение качества реконструкции реальных изображений по сравнению с существующими алгоритмами: \linebreak StyleGAN2 Projector и ALAE. 
Также  результаты экспериментов показывают улучшение реалистичности изображений и контроля над признаками при переносе лицевых признаков по сравнению с существующими алгоритмами: Image2StyleGAN и GANSpace, но при этом демонстрируют увеличение времени, затрачиваемого на обработку изображений.

\end{itemize}