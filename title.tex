%% Если что-то забыли, при компиляции будут ошибки Undefined control sequence \my@title@<что забыли>@ru
%% Если англоязычная титульная страница не нужна, то ее можно просто удалить.
\filltitle{ru}{
    %%
    chair              = {Математическое обеспечение и администрирование информационных систем},
    %% Макрос filltitle ненавидит пустые строки, поэтому обязателен хотя бы символ комментария на строке
    %% Актуально всем.
    title              = {Использование латентного пространства генеративных нейронных сетей для семантического редактирования фотографий},
    % 
    %% Здесь указывается тип работы. Возможные значения:
    type               = {master},
    author             = {Сугоняев Андрей Дмитриевич},
    % 
    %% Актуально только для ВКР. Указывается код и название направления подготовки.
    %% Те, что с 03 в середине --- бакалавриат, с 04 --- магистратура.
    specialty          = {02.04.03 <<Математическое обеспечение и администрирование информационных систем>>},
    % 
    %% Актуально только для ВКР. Указывается шифр и название образовательной программы.
    %%   ВМ.5665.2019 <<Математическое обеспечение и администрирование информационных систем>>
    %% Шифр и название программы можно посмотреть в учебном плане, по которому вы учитесь. 
    %% СВ.* --- бакалавриат, ВМ.* --- магистратура. В конце --- год поступления (не обязательно ваш, если вы были в академе/вылетали).
    programme          = {ВМ.5006.2019 <<Математическое обеспечение и администрирование информационных систем>>},
    %
    supervisorPosition = {д.ф.-м.н., профессор}, % Терехов А.Н.
    supervisor         = {А.Н. Терехов},  
    % 
    consultantPosition = {генеральный директор ООО <<СКЗ>>},
    consultant         = {А.А. Пименов},
    %
    %reviewerPosition   = {должность ООО <<Место работы>> степень},
    reviewerPosition   = {главный специалист ООО <<СКЗ>>, к.т.н.},
    reviewer           = {С.И. Федоренко},
}

\filltitle{en}{
    chair              = {Software and Administration of Information Systems},
    title              = {Using latent space of generative adversarial network for semantic photo editing},
    type               = {master},
    author             = {Andrey Sugonyaev},
    % 
    specialty          = {02.04.03 ``Software and Administration of Information Systems''},
    % 
    programme          = {ВМ.5006.2017 ``Software and Administration of Information Systems''},
    %
    % 
    %% Note that common title translations are:
    %%   кандидат наук --- C.Sc. (NOT Ph.D.)
    %%   доктор ... наук --- Sc.D.
    %%   доцент --- docent (NOT assistant/associate prof.)
    %%   профессор --- prof.
    supervisorPosition = {Sc.D., prof.},
    supervisor         = {A.N. Terekhov},
    % 
    consultantPosition = {CEO ``CVS''},
    consultant         = {A.A. Pimenov},
    %
    %reviewerPosition   = {position at ``Company'', degree if present},
    reviewerPosition   = {Chief Specialist ``CVS'', C.Sc.},
    reviewer           = {S.I. Fedorenko},
}