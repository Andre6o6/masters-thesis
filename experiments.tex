% !TeX spellcheck = ru_RU

План:

\begin{enumerate}

\item 
Для количественной оценки результатов работы решения, а также сравнения с предложенных в решении алгоритмических модификаций с алгоритмами, указанными в главе Обзор, был проведен ряд экспериментов.

\item 
В ходе экспериментов были исследованы следующие вопросы:
\begin{itemize}
    \item 
    % Оценка качества реконструкции реальных изображений
    Насколько качественно производится реконструкция входных изображений в латентном пространстве генеративной состязательной сети с помощью примененного в решении алгоритма по сравнению с существующими state-of-the-art алгоритмами?
    \item 
    % Оценка качества переноса лицевых признаков
    Насколько хорошо работает примененный в решении алгоритм переноса признаков по сравнению с существующими state-of-the-art алгоритмами?
\end{itemize}

\item 
Следуя мотивации, представленной в литературе по генеративным состязательным сетям, были использованы следующие метрики:
\begin{itemize}
    \item 
    LPIPS \cite{zhang2018lpips}:
    \item
    Face identity score:
    \item 
    FID \cite{heusel2017fid}:
\end{itemize}


\end{enumerate}

\begin{center}
\begin{tabular}{ |c|c|c| } 
 \hline
 cell1 & cell2 & cell3 \\ 
 cell4 & cell5 & cell6 \\ 
 cell7 & cell8 & cell9 \\ 
 \hline
\end{tabular}
\end{center}


%RQ 1: Сравнение качества реконструкции производится со следующими методами(?) - Image2StyleGAN, StyleGAN2 Projector, ALAE. Для этого производится отображение изображений, взятых из датасета CelebA-HQ, в латентное пространство соответствующей модели и подсчитываются метрики визуальной схожести (LPIPS, Face identity score) между реальным изображением и его реконструкцией.
%(Для итеративных методов максимальное кол-во итераций ограничивается 200.)
%Результаты представлены в Табл. 1.
%(Анализ говорит, что энкодер-методы страдают от потери качества, при этом оптимизационные/итеративные позволяют улучшать изображение [тут нужен анализ сходимости], и предоставляют artistic control/time-accuracy tradeoff, гибридный же метод ускоряет оптимизацию и улучшает сходимость.)

%RQ 2: 
%(Анализ должен сказать, что мы выигрываем по качеству, контролю, и identity preservation, но проигрываю по времени)


%\subsection{Задачи эксперимента}
%\subsection{Данные}
%\subsection{Метрики}
%\subsection{Сравниваемые методы}
%\subsection{Анализ результатов}