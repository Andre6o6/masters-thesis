\input{header.tex}

\usepackage{caption}
\usepackage{listings}

\DeclareCaptionFont{white}{ \color{white} }
\DeclareCaptionFormat{listing}{
    \parbox{\textwidth}{\hspace{15pt}#1#2#3}
}
\captionsetup[lstlisting]{ format=listing
  %, labelfont=white, textfont=white
  , singlelinecheck=false, margin=0pt, font={bf}
}

\graphicspath{ {images/} }

\begin{document}
%% Если что-то забыли, при компиляции будут ошибки Undefined control sequence \my@title@<что забыли>@ru
%% Если англоязычная титульная страница не нужна, то ее можно просто удалить.
\filltitle{ru}{
    %%
    chair              = {Математическое обеспечение и администрирование информационных систем},
    %% Макрос filltitle ненавидит пустые строки, поэтому обязателен хотя бы символ комментария на строке
    %% Актуально всем.
    title              = {Использование латентного пространства генеративных нейронных сетей для семантического редактирования фотографий},
    % 
    %% Здесь указывается тип работы. Возможные значения:
    type               = {master},
    author             = {Сугоняев Андрей Дмитриевич},
    % 
    %% Актуально только для ВКР. Указывается код и название направления подготовки.
    %% Те, что с 03 в середине --- бакалавриат, с 04 --- магистратура.
    specialty          = {02.04.03 <<Математическое обеспечение и администрирование информационных систем>>},
    % 
    %% Актуально только для ВКР. Указывается шифр и название образовательной программы.
    %%   ВМ.5665.2019 <<Математическое обеспечение и администрирование информационных систем>>
    %% Шифр и название программы можно посмотреть в учебном плане, по которому вы учитесь. 
    %% СВ.* --- бакалавриат, ВМ.* --- магистратура. В конце --- год поступления (не обязательно ваш, если вы были в академе/вылетали).
    programme          = {ВМ.5006.2019 <<Математическое обеспечение и администрирование информационных систем>>},
    %
    supervisorPosition = {д.ф.-м.н., проф.}, % Терехов А.Н.
    supervisor         = {А.Н. Терехов},  
    % 
    consultantPosition = {генеральный директор ООО <<СКЗ>>},
    consultant         = {А.А. Пименов},
    %
    reviewerPosition   = {должность ООО <<Место работы>> степень},
    reviewer           = {Р.Р. Рецензент},
}

\filltitle{en}{
    chair              = {Information Systems Administration and Mathematical Support},
    title              = {Using latent space of generative adversarial network for semantic photo editing},
    type               = {master},
    author             = {Andrey Sugonyaev},
    % 
    specialty          = {02.04.03 ``Software and Administration of Information Systems''},
    % 
    programme          = {ВМ.5006.2017 ``Software and Administration of Information Systems''},
    %
    % 
    %% Note that common title translations are:
    %%   кандидат наук --- C.Sc. (NOT Ph.D.)
    %%   доктор ... наук --- Sc.D.
    %%   доцент --- docent (NOT assistant/associate prof.)
    %%   профессор --- prof.
    supervisorPosition = {Sc.D, prof.},
    supervisor         = {A.N. Terekhov},
    % 
    consultantPosition = {General Director ``CVS''},
    consultant         = {A.A. Pimenov},
    %
    reviewerPosition   = {position at ``Company'', degree if present},
    reviewer           = {R.R. Reviewer},
}
\maketitle
\setcounter{tocdepth}{2}
\tableofcontents

\section*{Введение}
% !TeX spellcheck = ru_RU

Огромное количество информации об окружающем мире человеческий мозг воспринимает интуитивно. 
От классификации объектов среди десятков тысяч возможных вариантов, до распознавания трехмерных сцен, детектирования движения и взаимодействий между объектами —-- человеческий мозг способен извлечь и проанализировать эту информацию за доли секунд \cite{dicarlo2012brain}. 
Построение компьютерных алгоритмов, способных приблизиться по производительности и гибкости к механизмам распознавания в человеческом мозге является одной из первостепенных задач области под названием Computer Vision.

Эффективным инструментом извлечения и анализа такой информации являются генеративные модели \cite{jebara2012machine}. 
Их основная идея отражена в известной цитате Ричарда Фейнмана: “Я не понимаю то, что я не могу воссоздать”. 
Эти модели нацелены на моделирование наблюдаемых данных, выявление их внутренней структуры и использование этой структуры для генерации новых данных.

Среди генеративных моделей, используемых в Computer Vision, на сегодняшний день генеративные состязательные сети показывают  лучшие результаты \cite{generative2020survey}. 
Генеративные состязательные сети \cite{goodfellow2014generative} формулируют процесс обучения генеративной модели в виде антагонистической игры между двумя игроками --- генератором и дискриминатором. 
Генератор является нейронной сетью, задача которой состоит в генерировании синтетических данных, как можно более похожих на реальные. 
Дискриминатор также является нейронной сетью, задача которой состоит в том, чтобы как можно более точно отличать синтетические данные, полученные генератором, от реальных данных, взятых из обучающей выборки. 
Процесс совместного конкурентного обучения этих сетей продолжается вплоть до достижения системой равновесия по Нэшу \cite{goodfellow2017nips}, в котором генератор способен генерировать изображения, неотличимые дискриминатором от реальных.

Особенно большие успехи были достигнуты генеративными состязательными сетями в задаче генерации изображений лиц \cite{progressive-growing-gan, StyleGAN, karras2020stylegan2}. 
К этой задаче проявляется особый интерес, т.к. она хорошо демонстрирует возможности генеративных состязательных сетей по кодированию визуальной информации. 
Ведь известно, что лица обладают большим количеством факторов вариации, которые легко и интуитивно распознаются человеком, но при этом трудно поддаются анализу машинными алгоритмами \cite{face2008perception}.
Генеративные состязательные сети способны упаковать информацию о факторах вариации в виде вектора небольшой длины, а затем восстановить из этого вектора исходное изображение, сохраняя  высокую реалистичность и разнообразие признаков. 
Стоит также отметить, что последние нейробиологические исследования указывают на наличие схожих механизмов кодирования информации о лицевых признаках в мозге человека \cite{facial2017code}.

Существующие генеративные состязательные сети способны генерировать реалистичные изображения лиц высокого разрешения, т.к. они обучились эффективно кодировать информацию о признаках (другими словами, \emph{семантику}). 
Можно предположить, что такое кодирование позволит генеративным состязательным сетям также хорошо справляться с задачей редактирования изображений в терминах признаков лиц, которые интуитивно понятны человеку (далее --- \emph{задача семантического редактирования}). 
Тем не менее, существующие подходы, которые применяют генеративные состязательные сети для решения этой задачи, не могут оперировать всем многообразием признаков, которое доступно при генерации изображений. 
Большинство таких подходов ограничивается редактированием единственного признака, а для работы с другими признаками они требуют полного переобучения \cite{cycle2017gan, residual2017gan}. 
В других случая они  модифицируют  архитектуру генеративной состязательной сети, позволяя задавать лишь определенное значение для фиксированного набора признаков \cite{choi2018stargan, liu2019stgan}. 
А это, в свою очередь, не позволяет им в полной мере использовать извлекаемую при обучении семантику изображений, а также усложняет внедрение новых техник и улучшений, появляющихся в области генерации изображений.

В данной работе предлагается решить задачу семантического редактирования с помощью генеративных состязательных сетей, взяв в качестве основы предобученную сеть для генерации лиц и модифицировав ее архитектуру для предоставления возможности манипулировать ее внутренними представлениями.


\section*{Постановка задачи}
\label{sec:task}
% !TeX spellcheck = ru_RU
Целью данной работы является разработка и реализация решения для семантического редактирования изображений лиц на основе генеративных состязательные сетей. 
Решение должно быть способно определить то, как семантическая информация о лицевых признаках закодирована в латентном пространстве генеративной состязательной сети, обученной для генерации изображений лиц, и использовать эту информацию для манипулирования данными признаками.

Для достижения цели были поставлены следующие задачи:

\begin{enumerate}
\item провести обзор существующих архитектур генеративных состязательных сетей, предназначенных для генерации изображений лиц, а также существующих алгоритмов выделения семантики изображения;
\item разработать архитектуру решения для семантического редактирования изображений  на основе состязательных генеративных сетей;
\item реализовать прототип решения;
\item произвести эксперименты для качественной и количественной оценки полученного решения.
\end{enumerate}


\section{Обзор}
\label{sec:relatedworks}
\subsection{Генеративные состязательные сети}
% https://arxiv.org/pdf/2004.02546.pdf -- see Background for decent architecture explanation... 

Генеративные состязательные сети (\emph{Generative adversarial network}, сокр., \emph{GAN})  это семейство генеративных моделей, разработанное Яном Гудфеллоу в 2014 году \cite{goodfellow2014generative}.  Они формулируют процесс обучения  генеративной модели в виде антагонистической игры между двумя игроками: \emph{генератором} и \emph{дискриминатором}.

Генератор $G$ представляет собой глубокую нейронную сеть, задающую отбражение из латентного пространства $Z$ в пространство данных $\mathcal X$.
Принимая на вход случайные вектора из некоторого заданного распределения $p(z)$ в $Z$, он производит набор синтетических данных, пытаясь аппроксимировать распределение обучающей выборки $q_{data}(x)$.

Дискриминатор $D$ представляет собой глубокую нейронную сеть, задающую отбражение из пространства данных $\mathcal X$ в $\mathbb R$. Получив на вход некоторый элемент из пространства $\mathcal X$, дискриминатор ~возвращает~(характеризует) вероятность того, что он был получен из распределения реальных данных $q_{data}(x)$, а не сгенерирован генератором $G$.

\begin{figure}[h]
\begin{center}
    \includegraphics[width=0.9\textwidth]{gan}
    \caption{Архитектура генеративной состязательной сети (написать откуда взял)}
    \label{fig:subim11}
\end{center}
\end{figure}

Роль дискриминатора состоит в том, чтобы как можно более точно отличать синтетические данные, полученные генератором, от реальных данный, взятых из обучающей выборки, в то время как генератор пытается обмануть дискриминатор путем генерирования синтетических данных, как можно более похожих на реальные.
Процесс совместного конкурентного обучения продолжается вплоть до достижения парой $(G, D)$ <<седловой точки>> (т.е. равновесия по Нэшу) \cite{goodfellow2017nips} функции выигрыша
$$
\min_{G} \max_{D} V(G, D) = \mathop{\mathbb{E}}_{x \sim q_{data}(x)} [\log D(x)] + \mathop{\mathbb{E}}_{z \sim p(z)} [\log (1 - D(G(z)))] ,
$$
в которой генератор способен генерировать данные, неотличимые дискриминатором от реальных.

\subsection{Генеративные нейронные сети для генерации лиц}

При решении задачи генерации изображений (в.т.ч. изображений лиц), в качестве дискриминатора используется сверточная нейронная сеть [VGG / conv ?], а в качестве генератора --- сверточная нейронная сеть со слоями транспонированной свертки [DCGAN].
Пространство данных $\mathcal X$ в данном случае предстваляет собой пространство изображений, каждый элемент которого обладает некоторым набором семантических признаков, например в случае лиц --- положение головы, пол, возраст.

\paragraph{BigGAN.}
Генерация изображений высокого разрешения довольно долго была трудной(непосильной) задачей для GAN-ов. Она требует генеративной модели с большим количеством параметров, что, в случае GAN-ов, сказывается на стабильности обучения [че то типа Why GANs are hard to train].

% ... разработанная компанией DeepMind.
(Генеративная состязательная сеть) BigGAN \cite{bigGAN} являлась ?первым шагом к решению данной проблемы. Эта сеть спроектирована с целью масштабирования архитектуры для генерации изображений высокого разрешения.
Она внесла ряд модификаций в стандартную архитектуру генеративных состязательных сетей (и процесс генерации), таких как self-attention слой, спектральная нормализация весов (, Truncation Trick). Это позволило увеличить количество весов модели в 2 раза и добиться качественной генерации изображений с разрешением до $512\times512$ на самых разнообразных классах изображений.


\paragraph{StyleGAN.}
StyleGAN \cite{StyleGAN} – это генеративная состязательная сеть, архитектура которой разрабатывалась специально с целью генерации реалистичных изображений лиц.
Одной из сложностей генерации изображений является кривизна, или запутaнность (\emph{entanglement}), латентного пространства. 
Входное латентное пространство $\mathcal Z$ должно подстраиваться под вероятносное распределение тренировочных данных, а потому любые сдвиги/неточности, например отсутствие в тренировочных данных комбинации нескольких признаков (Рис. \ref{fig:stylegan-mapping}), приводят к искажениям.
В результате латентное пространство (приходится рассматривать) не как линейное евклидово пространство, а как искривленное пространство \cite{arvanitidis2018oddity}.

% Для решения этой проблемы...
Одной из ключевых особенностей архитекуры StyleGAN является наличие промежуточного латентного пространства $\mathcal W$. Во время обучения StyleGAN выучивает нелинейное преобразование $f: \mathcal Z \mapsto \mathcal W$, что позволяет избавиться от ограничений, накладываемых вероятносным распределением тренировочных данных, и получить более линейное латентное пространство.

% реализует архитектуру ...?
StyleGAN предлагает альтернативную архитектуру генератора, которая основывается на идеях из style transfer. 
Синтез изображения начинается с фиксированного входного вектора, а информация о латентном представлениии последовательно встраивается в каждый слой генератора, начиная с первых слоев генератора с пространственной размерностью $4\times4$ и заканчивая последними слоями с размерностью $1024\times1024$.
Это позволяет контролировать силу проявления различных семантических признаков изображения на разных ?масштабах, от грубых признаков до тонких деталей.
В сочитании с встраивании шума напрямую в слои сети, такая архитектура позволяет отделить высокоуровневые атрибуты изображения (положения лица, личность человека) от случайных вариационных факторов (волосы, веснушки и т.п.).

\begin{figure}[h]
\begin{center}
    \includegraphics[width=0.7\textwidth]{stylegan_mapping}
    \caption{Иллюстрация действия промежуточного латентного пространства на примере двух факторов вариации. Изображение взято из \cite{StyleGAN}. \emph{Расписать a, b, c.}}
    \label{fig:stylegan-mapping}
\end{center}
\end{figure}
% Illustrative example with two factors of variation (image features, e.g., masculinity and hair length).  (a) An example training set where some combination (e.g., long haired males) is missing. (b) This forces the mapping from Z to image features to become curved so that the forbidden combination disappears in Z to prevent the sampling of invalid combinations.  (c) The learned mapping from Z to W is able to “undo” much of the warping.


\paragraph{ALAE.}
% We designed an AE architecture where we allow the latent distribution to be learned from data to address entanglement (A). The output data distribution is learned with an adversarial strategy (B). Finally, to implement (A) and (B) we impose the AE reciprocity in the latent space (C).

% декомпозируем генератор и дискриминатор ... - но это не очень понятно
ALAE \cite{ALAE} – это генеративная нейронная сеть, которая совмещает в себе особенности генеративных состязательных сетей и вариационных автоэнкодеров [].

Она модифицирует стандартную архитектуру генеративных состязательных сетей путем добавления \emph{перед} генератором и \emph{перед} дискриминатором некоторых выучваемых отображений в промежуточное латентное пространство.
% наложение на промежуточное латентное пространство ограничения о взаимной близости представлений дает возможность выучить latent distribution с помощъю autoencoder стратегии, и data distribution с помощью состязательной стратегии.
Из-за этого он жертвует качеством генерации, но дополнительно выучивает обратное отображение в латентное пространство сети.


\subsection{Алгоритмы отображения изображений в латентное пространство сети}
% TODO выровнять алгоритмы/методы/подходы
Для работы с реальными изображениями генеративным состязательным сетям требуется обратное отображение, позволяющее по входному изображению получить соответствующий ему латентный вектор. 
Классическая архитектура генеративных состязательных сетей не задает такого отображения напрямую, а потому для работы с реальными изображениями реализуются алгоритмы, позволяющие аппроксимировать данный латентный вектор.

%% \subsubsection{Обучение дополнительного энкодера}
Одним из самых распространенных является обучение энкодера \cite{donahue2016adversarial} --- дополнительной нейронной сети, которая будет отображать входное изображение в латентное пространство заданной генеративной состязательной сети.
% датасет --> набор данных
Энкодер имеет сверточную архитектуру, и процесс его обучения сводится к задаче регрессии латентного вектора по заданному входному изображению. 
Обучение энкодера производится на датасете пар <\emph{изображение, вектор}>, состоящем из набора синтетических изображений, сгенерированных генеративной состязательной сетью, и входных векторов, с помощью которых изображение было получено.

%% \subsubsection{Латентная оптимизация}
Латентная оптимизация \cite{perarnau2016invertible} --- это алгоритм приближения латентного вектора, который заключается в нахождении латентного вектора путем минимизации некоторой заданной функции потерь реконструкции.
% Этот алгоритм задает функцию потерь $ f(x_real, G(z)) $, которая показывает, насколько сгенерированное изображение близко к входному.
% Примером такой функции потерь может служить среднеквадратичная ошибка в пространстве пикселей (pixel-wise), или perpetual loss, заданный в пространстве признаков, извлеченных сетью VGG[]. 
% https://arxiv.org/abs/2002.04185 -- Smoothness and Stability in GANs
% Кратко описать условия, при которых GD сойдется
% Under the assumption of generator being smooth (which, since generator is a composition of linear maps and activation functions, depends only on smoothness of activations), we can use gradient descent to minimize reconstruction loss.
% Поскольку генератор является отображением, дискпвтпоп по входному вектору, а его гладкость зависит главным образом от гладкости функций активации [], это позволяет минимизировать ф.п. с помощью г.с., итеративно обновляя латентный вектор.
Генератор сети GAN является композицией линейных слоев и функций активации, а потому при использовании гладких функций активации также является гладким отображением.
Это позволяет применить градиентный спуск и метод обратного распространения ошибки для нахождения латентного вектора, минимизирующего заданную функцию потерь реконструкции.

% Использование арзитектуры StyleGAN позволяет еще больше улучшить процесс латентной оптимизации. Оптимизация в промежуточном латентном пространства W, в меньшей степени подверженного искривлению, позволяет стабилизировать градиентный спуск [Image2StyleGAN ?]. Кроме того, архитектура StyleGAN устроена таким образом, что вместо последовательных перобразований над латентным вектором, латентный вектор подается на вход каждому слою генератора in parallel fashion. Это позволяет латентный вектор каждого латентного слоя оптимизировать отдельно, тем самым получив оптимизацию в раширенном промежуточном пространстве W+. В результате такой оптимизации можно получить изображения, которые невозможно было бы сгенерировать в стандартном pipeline. [What GANs cannot generate, Cheese theorem].


\subsection{Алгоритмы выделения семантик изображения в латентном пространстве сети}
Идея манипулирования латентным вектором в латентном пространстве генеративной состязательной сети основана на наблюдении о том, что к латентным векторам применима векторная арифметика \cite{radford2015unsupervised}.
% Из этого возникает идея нахождения преобразований в латентном пространстве, которые соответствовали бы изменениям семантических признаков на генерируемом изображениии.
% Большинство (?) исследований (св-в латентного пространства) основываются на пердположении о линейности латентного пр-ва / о том, что отдельным факторам вариации генерируемых изображений (изображений из тренировочной выборки) соответствует некоторое линейное подпространство в латентном пр-ва.
% Линейный сдвиг в направлении ...
% (Нужно конкретно определить, что мы ищем, т.е. important directions / semantics -- найти для этого русское определение).

% Далее говорим про GANSpace, котороый использует метод главных компонент для выделения значимых направлений. Требуется дальнейший визуальный анализ, чтобы определить, каким изменениям/семантикам они соответствуют.
Другой подход \cite{hrknen2020ganspace} использует метод главных компонент для уменьшения размерности латентного пространства и выделения наиболее важных направлений. Дальнейший визуальный/качественный анализ полученных направлений показывает, что они соответствуют значимым изменениям в изображении.

% Для более качественного исследования ?направлений? в латентном пространстве InterFaceGAN предлагает находить направления путем нахождения/фиттинга оптимальной разделяющей гиперплоскости. 
Один из существующих подходов \cite{shen2020interfacegan} выдвигает предположение о линейности латентного пространства, и исследует его с целью нахождения векторов, соответствующих определенным заданным изменениям в признаках изображения.


\section{Алгоритм выделения линейных подпространств, соответствующих семантикам изображения}
\emph{Здесь будет расписан алгоритм выделения семантик, лежащий в основе решения.}

Для выделения семантик изображения принимается предположение о линейности латентного пространства.

Рассмотрим произвольный бинарный признак, заданный дискриминативной функцией $f : \mathcal X \mapsto \{0,1\}$, определяющей наличие или отсутствие этого признака на изображении.
При справедливости этого предположения линейная интерполяция между двумя латентными векторами, соответствующими двум изображениям, на одном из которых признак присутствует, а на другом --- отсутствует, приведет к постепенному и непрерывному изменению этого признака на генерируемом изображении. 

Как показано в \cite{StyleGAN}, линейная разделимость бинарных признаков в промежуточном латентном пространстве $\mathcal{W}$ позволяет найти векторы направлений, соответствующих отдельным факторам вариации, т.е. отдельным признакам.

Для нахождения данных направлений используется следующий метод.
\begin{itemize}
    \item В качестве функции $f$ используется нейросетевой классификатор, обученный на датасете CelebA-HQ. Архитектура крассификатора аналогична архитектуре дискриминатора, используемого в \cite{progressive-growing-gan, StyleGAN}. В качестве рассматриваемых признаков были выбраны \emph{наличие улыбки} и \emph{поворот головы}.
    \item С помощью имеющегося генератора $G$ генерируется набор данных, состоящий из $20000$ изображений. Данный набор размечается с помощью классификатора $f$.
    \item На полученном наборе данных обучается линейная SVM \cite{svm} и находится оптимальная разделяющая гиперплоскость. 
\end{itemize}
Нормальный вектор полученной гиперплоскости задает направление, соответствующее рассматриваемому признаку.
%\subsection{Генерация данных}
%\subsection{Оптимальная разделяющая гиперплоскость}
%\subsection{Линейная разделимость в латентном пространстве}

\begin{figure}[h]
\begin{center}
    \includegraphics[width=0.7\textwidth]{boundary-SVM}
    \caption{Иллюстрация процесса нахождения векторов направлений, соответствующих отдельным факторам вариации, путем нахожения оптимальной разделяющей плоскости.}
    \label{fig:svm-boundary}
\end{center}
\end{figure}

\section{Архитектура решения}
% !TeX spellcheck = ru_RU

\begin{figure}[h]
\begin{center}
    \includegraphics[width=0.9\textwidth]{Architecture_eng}
    \caption{Архитектура решения.}
    \label{fig:architecture}
\end{center}
\end{figure}


В рамках данной работы было разработано решение для семантического редактирования изображений, позволяющее редактировать изображения лиц в терминах высокоуровневых лицевых признаков, таких как положение головы, наличие улыбки и т.д.
По заданному бинарному лицевому признаку, это решение способно выделить семантический вектор в латентном пространстве генеративной состязательной сети, и произвести перенос информации об этом признаке с изображения-образца на целевое изображение.

Решение состоит из двух функциональных частей: модуля реконструкции изображений (рис. \ref{fig:architecture}, \emph{Image reconstruction module}), и модуля семантического редактирования (рис. \ref{fig:architecture}, \emph{Feature transfer module}).

\subsection{Модуль реконструкции изображений}
Модуль реконструкции изображений осуществляет отображение входных изображений в латентное пространство имеющейся генеративной состязательной сети.
Он реализует гибридный алгоритм отображения, совмещающий обучение энкодера и латентную оптимизацию. 
Энкодер (компонент \emph{Initial predictor} на рис. \ref{fig:architecture}) используется для получения грубого начального приближения.
Затем это приближение итеративно улучшается с помощью латентной оптимизации (компонент \emph{Latent optimizer} на рис. \ref{fig:architecture}).

Такой алгоритм предостравляет пользователю гибкий контроль баланса между точностью реконструкции изображения и затраченным на обработку временем.

В качестве используемой архитектуры генеративной состязательной сети выбрана сеть StyleGAN, что позволет производить оптимизацию в расширенном латентном пространстве $\mathcal W+$.

\subsection{Модуль семантического редактирования}
Модуль семантического редактирования осуществляет семантическое редактирование изображений, полученных в результате реконструкции входных изображений в латентном пространстве генеративной состязательной сети. 

Пользователю предоставляется возможность выбрать в качестве редактируемого признака один из признаков, для которых семантические вектора были вычислены заранее.
Альтернативно, данный модуль предоставляет возможность извлечь семантический вектор для произвольного бинарного лицевого признака.
Для этого модулем реализуется алгоритм InterFaceGAN (компонент \emph{Semantic vector extraction} на рис. \ref{fig:architecture}).

Найденный семантический вектор передается в компонент, реализующий алгоритм переноса признака с изображения-образца на целевое изображение (\emph{Feature transfer} на рис. \ref{fig:architecture}).
Этот алгоритм основывается на алгоритме линейного сдвига латентного вектора, описанном в \cite{abdal2019image2stylegan}.
Для того, чтобы добиться более качественного переноса редактируемого признака и сохранения остальных признаков, производится модификация данного алгоритма, позволяющая производить нелинейный сдвиг в латентном пространстве сети.

Перенос признака сводится к минимизации функции потерь переноса
$$ 
\min_{\mathbf w} L(\mathbf w) = L_{feature}(\mathbf w, \mathbf w_{exemplar}) + \alpha~L_{identity}(\mathbf w, I_{real}),
$$
где член $L_{feature}$ осуществляет линейный сдвиг латентного вектора, найденного в ходе реконструкции входного изображения, в направлении, заданном семантическим вектором признака;  член $~L_{identity}$ осуществляет корректировку сдвига таким образом, чтобы он изменял лишь рассматриваемый признак, и, в частности, не затрагивал признаки, определяющие личность изображенного человека.


\section{Особенности реализации}
\emph{Здесь описывается реализация, в.т.ч. гиперпараметры и процесс обучения доп. классификаторов.}

\subsection{Энкодер начального приближения латентной оптимизации}
%Возможно этот абзац стоит в переработанном виде кинуть в архитектуру!
Чтобы ускорить процесс сходимости латентной оптимизации, в данной работе предлагается использовать гибридный подход. Предлагается обучить дополнительную нейронную сеть-энкодер, которая по входному изображению даст грубое приближение его латентного вектора. Использование данное приближение в качестве начального приближения позволит ускорить роцесс сходимости.

Для предсказания начального приближения латентной оптимизации была обучена сверточная нейронная сеть ResNet.
%почему резнет

Для ее обучения с помощью имеющейся генеративной состязательной сети было сгенерированно $50000$ изображений и соответствующих им латентных векторов.
По входному изображению сеть ResNet обучалась предсказывать его латентный вектор. В качестве функции потерь использован логарифм гиперболического синуса ошибки (\emph{Log-Cosh Loss} \cite{chen2019log}), который является гладким аналогом средней абсолютной ошибки.
%log cosh loss, то что он хорош при обучении вариационных автокодировщиков

Нейронная сеть реализована на  фреймворке PyTorch. Сеть обучалать $20$ эпох методом обратного распространения ошибки с использованием оптимизационного алгоритма Adam.
%график нужен

\subsection{Нейронная сеть для латентной оптимизации}

Генератор $G$ является дифференциируемой по входам нейронной сетью, что позволяет напрямую оптимизировать латентный вектор, минимизируя функцию потерь реконструкции. Этот процесс называется латентной оптимизацией. 

\emph{Нужно поправить обозначения на изображении.}
\begin{figure}[h]
\begin{center}
    \includegraphics[width=0.9\textwidth]{optim_pipeline_ru}
    \caption{Архитектура нейронной сети для латентной оптимизации}
    \label{fig:optim_pipelin}
\end{center}
\end{figure}

В качестве функцию потерь реконструкции используется взвешенная сумма среднеквадратичной ошибки в пространстве пикселей и визуальной функции потерь (\emph{perceptual loss} \cite{Johnson2016Perceptual}).
%Здесь нужно сказать, что следуем Image2StyleGAN, что используем W+

\emph{Дальше нужно написать достаточно сложный абзац, объясняющий пространство $W+$ и оптимизацию отдельных грубых мап признаков.}
%собственно здесь описываем модификацию того, что ступенчато оптимизируемся и что это позволяет не улететь датеко от данных (см. Latent Space Oddity).


%\subsection{Алгоритм выделения семантик}
\subsection{Алгоритм выделения линейных подпространств, соответствующих семантикам изображения}
\emph{Здесь будет расписан алгоритм выделения семантик, лежащий в основе решения.}

Для выделения семантик изображения принимается предположение о линейности латентного пространства.

Рассмотрим произвольный бинарный признак, заданный дискриминативной функцией $f : \mathcal X \mapsto \{0,1\}$, определяющей наличие или отсутствие этого признака на изображении.
При справедливости этого предположения линейная интерполяция между двумя латентными векторами, соответствующими двум изображениям, на одном из которых признак присутствует, а на другом --- отсутствует, приведет к постепенному и непрерывному изменению этого признака на генерируемом изображении. 

Как показано в \cite{StyleGAN}, линейная разделимость бинарных признаков в промежуточном латентном пространстве $\mathcal{W}$ позволяет найти векторы направлений, соответствующих отдельным факторам вариации, т.е. отдельным признакам.

Для нахождения данных направлений используется следующий метод.
\begin{itemize}
    \item В качестве функции $f$ используется нейросетевой классификатор, обученный на датасете CelebA-HQ. Архитектура крассификатора аналогична архитектуре дискриминатора, используемого в \cite{progressive-growing-gan, StyleGAN}. В качестве рассматриваемых признаков были выбраны \emph{наличие улыбки} и \emph{поворот головы}.
    \item С помощью имеющегося генератора $G$ генерируется набор данных, состоящий из $20000$ изображений. Данный набор размечается с помощью классификатора $f$.
    \item На полученном наборе данных обучается линейная SVM \cite{svm} и находится оптимальная разделяющая гиперплоскость. 
\end{itemize}
Нормальный вектор полученной гиперплоскости задает направление, соответствующее рассматриваемому признаку.
%\subsection{Генерация данных}
%\subsection{Оптимальная разделяющая гиперплоскость}
%\subsection{Линейная разделимость в латентном пространстве}

\begin{figure}[h]
\begin{center}
    \includegraphics[width=0.7\textwidth]{boundary-SVM}
    \caption{Иллюстрация процесса нахождения векторов направлений, соответствующих отдельным факторам вариации, путем нахожения оптимальной разделяющей плоскости.}
    \label{fig:svm-boundary}
\end{center}
\end{figure}


\subsection{Алгоритм переноса лицевых признаков с изображения образца}

%Переформулировать в терминах оптимизации.

% Мы оптимизируем пока не сойдемся по одному Degree of Freedom, и при этом делаем ортогональный шаг по остальным Degree of Freedom.
На генерируемом изображении $g(\mathbf x + \alpha \mathbf n)$, полученном путем линейного сдвига в направлении полученного вектора $\mathbf n$, выбранный семантический признак будет более выражен при $\alpha > 0$, и менее выражен при $\alpha  < 0$.

%собственно здесь описываем модификацию того, что используется Face Model для введения нелинейности и сохранения личности.

\section{Эксперименты}
%Для того, чтобы проверить, насколько реализованное решение соответствует требованиям, а также для того, чтобы среди аналогичных вариантов реализации алгоритма распознавания P300 выбрать тот, который дает наилучший результат на реальных данных, потребовалось произвести ряд экспериментов.

\subsection{Условия эксперимента}
Эксперименты проводились на сервисе облачных вычислений Google Colab.

Входные данные, на которых проверяется подход:
\begin{itemize}
\item Датасет CelebA-HQ
\item Датасет FEI Face Database
\end{itemize}


\subsection{Метрики}
%Как мы сравниваем, что результаты двух подходов лучше или хуже
\begin{itemize}
    \item LPIPS \cite{zhang2018lpips}
    \item Face Identity score \cite{deng2018arcface}
    \item FID \cite{heusel2017fid}
    \item Attribute score
\end{itemize}
% FID score:To measure the diversity and quality of the output samples, we use the FID score between the test images and the edited images. We evaluate the results with 1k generated samples from the StyleGAN2 framework
% Face identity score:To measure the quality of the edit and quantify the identity preservation of the edits,  we evaluate the edited images using a face identity score.  We take a state-of-the-art classifier model for face recognition (Geitgey 2020) to output embeddings of the images. Given a pair of images, before and after edits, we calculate the Euclidean distance and the cosine similarity between the embeddings. Note that we use a different classifier from the attribute estimator used in training our StyleFlow. We choose three major edits for this purpose: light, pose, and expression.

\subsection{Исследовательские вопросы}
%Надо сформулировать то, чего мы хотели бы добиться работой (2 штуки будет хорошо):
\begin{itemize}
\item Оценка качества реконструкции реальных изображений
\item Оценка качества переноса лицевых признаков
%\item Хотим алгоритм, который лучше вот таких-то остальных
%\item Если в подходе можно включать/выключать составляющие, то насколько %существенно каждая составляющая влияет на улучшения
%\item Если у нас строится приближение каких-то штук, то на сколько точными %будут эти приближения
%\item и т.п.
\end{itemize}

% Первым исстедуемым вопросом являетс вопрос выбора (базовой)генеративной состязательной сети, которая будет (в основу) решения. [Сказать/повторить требования] Важно качество реконструкции реального изображения
% ХЗ, на самом деле это можно объяснить и без эксперимента, он будет только мешаться.

\subsection{Результаты}
Здесь будут таблицы и графики

%Пояснения по Research Questions.
%\subsubsection{RQ1} Пояснения
%\subsubsection{RQ2} Пояснения

%\subsection{Обсуждение результатов}
%Чуть более неформальное обсуждение, то, что сделано. Например, почему метод работает лучше остальных? Или, что делать со случаями, когда метод классифицирует вход некорректно.

%\section{Угрозы нарушения корректности (опциональный)}
%Если основная заслуга метода, это то, что он дает лучшие цифры, то стоит сказать, где мы могли облажаться, когда проводили численные замеры. 

\section*{Заключение}
% !TeX spellcheck = ru_RU

В рамках данной выпускной квалификационной работы разработано решение для редактирования лицевых признаков на изображении, использующее предобученную для генерации изображений лиц генеративную состязательную сеть StyleGAN. 

В ходе работы были получены перечисленные ниже результаты.

\begin{itemize}
\item Проведен обзор существующих архитектур генеративных состязательных сетей для генерации изображений лиц --— BigGAN, StyleGAN, ALAE; приведены доводы в пользу использования архитектуры StyleGAN в качестве основы для решения.
\item Разработана архитектура решения для семантического редактирования изображений, позволяющая произвести перенос произвольного бинарного лицевого признака с изображения-образца на целевое изображение.
\item Реализован прототип решения на языке Python с использованием фреймворка PyTorch. Исходные коды выложены на GitHub\footnote{https://github.com/Andre6o6/stylegan-editing}.

\item Проведены эксперименты для оценки работы решения. 
Результаты показывают улучшение качества реконструкции реальных изображений по сравнению с существующими алгоритмами: \foreignlanguage{english}{StyleGAN2 Projector} и ALAE. 
Также  результаты экспериментов показывают улучшение реалистичности изображений и контроля над признаками при переносе лицевых признаков по сравнению с существующими алгоритмами: GANSpace и InterFaceGAN, но при этом демонстрируют увеличение времени, затрачиваемого на обработку изображений.

\end{itemize}

% \nocite{*}
\setmonofont[Mapping=tex-text]{CMU Typewriter Text}
\bibliographystyle{ugost2008ls}
\bibliography{vkr}
\end{document}
